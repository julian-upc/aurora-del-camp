\documentclass[11pt,reqno]{amsart}

\usepackage{times}
\usepackage{microtype}
\usepackage{amssymb}
\usepackage{paralist}
\usepackage{url}
\usepackage{a4wide}
\usepackage{bbm}

\newcommand{\NN}{\mathbbm{N}}
\newcommand{\ZZ}{\mathbbm{Z}}

\newcommand{\lra}{\longrightarrow}

\numberwithin{equation}{section}
\newcommand{\ojo}[1]{\textsf{\bfseries\boldmath{#1}}}

\begin{document}

\renewcommand*\descriptionlabel[1]{%
\hspace\labelsep\normalfont\itshape #1:}

\newenvironment{mydesc}{%
  \begin{description}\setlength{\itemsep}{1ex}}%
  {\end{description}}

\title{Aurora del Camp -- specification}
\author{Julian Pfeifle}
\date{Version of \today}
\maketitle

This document outlines some of the issues that arise in the
implementation of Gilad's proposal~\cite{buzi11}.

\medskip
Given the availability of powerful free and open source solvers for
integer programs such as CBC~\cite{cbc}, it seems natural to pursue an
integer programming formulation. Of course, free solvers are not as
good as the best commercial ones, but the most recent
benchmarks~\cite{mittelmann11} indicate that CBC is reasonably
competitive; more precisely, it's the most competitive among all
solvers that have an open source license (in the case of CBC, the
``Eclipse Public License'') that permits Gilad to use it commercially
without paying any license fees.

\section{What will we implement?}

Even a small farm needs to plan many things:


\begin{description}

\item[Event scheduling] Determine the right sequence of activities, and the amount
  of crop to plant.

\item[Spatial distribution] Where will the crops be planted? Some constraints that come
  into the picture are crop rotation and spatial grouping of similar tasks for optimizing
  machine usage.

\item[Workforce administration] Given the individual characteristics of the participating
  workers, distribute the available work in the most efficient way. This will inevitably
  influence the event scheduling step.

\item[Resource administration] Take into account the non-renewable resources needed for
  operating the farm: amount of fertilizer, minerals, gasoline, seeds, etc.

\item[Profit maximization] This is one of the driving forces behind the objective function.

\end{description}

In the interest of rapid prototyping and starting the iterations, we start by implementing
only the event scheduling. Once the infrastructure for this is in place (web interface for
inputting new data and modifying existing input, scripting tools to generate the input
files for the solver, web interface for displaying the solution, etc.), we can think about
the rest.

In particular, in our first approximation for the scheduling problem, we will use
\emph{weeks} as the basic unit of time. However, in the meantime we might as well start to
collect data on the estimated duration in \emph{hours} of each task.


\section{Event-based modeling}


\subsection{Activities to be considered}

In the parlance of \cite{artigues-etal11}, we want to schedule a set
of \emph{activities} subject to certain constraints. 
%
% In our setup, each activity is associated to a certain specific part of the available
% fields. We refer to these parts or areas as \emph{lots}, and collect them into a
% set~$L$. We allow them to have different sizes.
%
They come in two types: $A=A_c\cup A_s$, where the activities in $A_c$ only affect the
field and are thus \emph{``common''} to all crops, and the activities~$A_s$ are
\emph{``specific''} to each crop. Moreover, we allow each activity to occur multiple times
in different time windows, to take into account repeated sowing, harvesting, etc. We keep
track of the repetitions by remembering, for each activity, a \emph{valid start time
  window} $\omega_{\rm start} = [w_1, w_2]$ of weeks in which it may begin. This has the
added advantage that certain characteristics of the activity, such as the duration or the
yield, may depend on the start time. For example, \emph{``planting tomatoes in week 34 or
  35''} and \emph{``planting tomatoes in week 48 or 49''} will be separate activities,
each with distinct durations and yield. The optimization process will determine whether to
carry out none, one or both of these activities. As an abstract notation, we write~$W$ for
the set of all useful start time windows $\omega = \omega_{\rm start}$, i.e., all
intervals contained in the planning horizon of the problem that are eligible for starting
a task.

\smallskip
Gilad breaks the two types of activities down as follows:

\begin{mydesc}
\item[Activities common to all crops] $ti$~for tilling, $rv$~for rotovating, $gm$~for
  green manure planting, $f\!t$~for fertilizing, $bb$~for bed building, $si$~for setting
  up irrigation, $sr$ for setting rows, $we$ for weeding. Since each of these activities
  can be repeated, we set
  \[
     A_c 
     \ = \
     \{ ti,\;rv,\;gm,\;f\!t,\;bb,\;si,\;sr,\;we\} \times W.
  \]
  A typical activity in $A_c$ is therefore $(ti,\omega)$ for some $\omega\in W$, which we
  write as $ti_{\omega}$ and take to mean ``till some unspecified portion of the field,
  starting in the time window~$\omega$''.


\item[Activities specific to a crop] $by$ for buying seeds, $ss$ for soaking seeds, $cs$
  for cutting or separating cloned seeds, $gc$~for false germination and cleaning,
  $pl$~for planting\footnote{We consider transplanting and planting to be the same
    process.}, $gr$ for growing, $f\!u$ for fumigating, $th$ for thinning, $tr$ for
  trimming, $co$ for covering, $ha$~for harvesting. We assemble these into
  \[
     A_s = \{ by,\;ss,\;cs,\;gc,\;pl,\;f\!u,\;th,\;tr,\;co,\;ha\}
     \times C \times W,
  \]
  where $C$ is the set of crops. A typical activity in~$A_s$ is therefore $(ha,c,
  \omega)=ha_{c, \omega}$, which means ``harvesting the crop $c\in C$ starting during the
  time window~$\omega$''.
\end{mydesc}

The estimates $|R|=5$, $|C|=40$ yield an upper bound of
\[
    8  \times 5 + 10 \times 40  \times 5
    \ = \
    2\,040
\]
activities in the model, which is a very manageable figure for both commercial and free
solvers.


\subsection{Precedence constraints}

We record the precedence constraints between  activities in a directed acyclic
graph~$H$. Thus, $t_1\lra t_2$ (also written $t_1<t_2$) is a directed edge
in~$H$ if $t_1$~must be completed before $t_2$~can start. 

\subsection{Chains}
We further group activities into \emph{chains}, or activities that must go
together. For example,
\[
   \chi
   \ = \
   \big(pl_{c,[w_0,w_1]}, \;
    we_{[w_2,w_3]},\;
    we_{[w_3,w_4]},\;
    gr_{[w_1,w_4]},\;
    ha_{c,[w_4,w_5]}\big)
\]
with $w_0<w_1<w_2<w_3<w_4<w_5$ is a chain consisting of \emph{planting}, \emph{weeding}
(twice), and \emph{harvesting} a crop $c$ that needs $w_4-w_1$~weeks to \emph{grow}. The
directed graph~$H$ records that in this particular chain, \emph{growing} and
\emph{weeding} may be carried out in parallel, but both these activities come after
\emph{planting} and before \emph{harvesting}. In general, the set of all chains is denoted
by~$K$, and if the first activity in a chain~$\chi$ is executed, all the others must be
executed too.



\subsection{Overview of the model}

We plan over several years. There will be an interface to put new activities into a
\emph{task queue} (chains of events that are not scheduled yet), and an interface to
record the actual progress of activities. This act of recording the real start and end
time of activities, environmental changes, changes in the available work force, etc., sets
certain variables in the problem formulation to fixed, known values, and allows Gilad to
frequently update the solution of the optimization problem and dynamically take into
account the latest developments.

\section{Scheduling}

\subsection{Variables}

We now proceed to model our problem.  To the activities in $A$ we associate a set~$E$ of
\emph{events}, which consist of the acts of starting and finishing each activity in~$A$;
thus, $|E|=2|A|=:n$. In consequence, we may consider $E=\{1,2,\dots, 2|A|=n\}$ to be
totally ordered.  Following \cite{artigues-etal11}, we introduce the following variables
and data:

\begin{enumerate}
\item A set of binary decision variables 
  \[
     Z
     \ = \
     \big\{z_{a,e}: a\in A, e\in E\big\},
  \]
  where each $z_{a,e}=1$ if and only if activity~$a$ starts at
  event~$e$ or is still in execution at event~$e$.

\smallskip
\item A set of continuous variables that indicate the starting time of each event:
  \[
     T
     \ = \
     \big\{t_e : e\in E\big\}
  \]

\smallskip
\item The set of continuous \emph{processing times} for each activity:
  \[
      \{p_a:a\in A\}
  \]  
  Since each activity knows about its starting time, we can easily take seasonal
  variability into account.

  \smallskip
% \item The total field space available, $S$, and the space taken up
%   by each activity,
%   \[
%       \{s_a : a\in A\}
%   \]

% \smallskip
\item The yield $y_{c,w}$ of each crop $c\in C$, depending on the week $w\in W$ in which
  it is planted.

\end{enumerate}

\subsection{Constraints internal to the model}

We now adapt the individual constraints from~\cite{artigues-etal11}:

\begin{mydesc}

\item[Not all activities have to execute] We do \emph{not} incorporate a constraint
  $\sum_{e\in E} z_{a,e}=1$ for all $a\in A$, because we do not wish to require all
  activities to execute. This leaves margin for chains to take place or not, but in a way
  that the decision whether or not it does is an outcome of the optimization process and
  not an a-priori input to the problem formulation.

\item[Activities in a chain must go together] The fact that either all or none of the
  activites in a given chain $\chi=(a_1,a_2,\dots,a_r)$ must be executed is expressed via
  the equality of the corresponding 0/1-variables:
  \begin{equation}
    a_1 \ = \ a_2 \ = \ \cdots \ = \ a_r.
  \end{equation}
  This property can be exploited to reduce the number of variables in the problem
  formulation, leaving in effect only one single binary variable for each chain. As yet,
  I'm not sure whether to do this simplification in the exposition, or leave it until
  implementation. 

\item[Setting the starting time] Instead of using $t_0=0$, we set
  \begin{equation}
     t_0
     \ = \
     w_0,
  \end{equation}
  where $w_0$ indexes the week of the year where optimization starts. In general, using
  weeks as units for time seems to be a good idea.

\item[Linearly ordering the execution start times] Since the events are supposed to be linearly
  ordered,  their execution times must also be:
  \begin{equation}
     t_e 
     \ \le \
     t_{e+1}
     \qquad\text{for all }
      e\in E\smallsetminus\{n\}
  \end{equation}

\item[Execution start constraints] Relations that implement start time windows:
  \begin{equation}
    w_{1,e}\ \le\ t_e \ \le \ w_{2,e}
    \qquad\text{for all } e\in E
  \end{equation}
  

\item[Duration constraints]
  \begin{equation}
     t_f 
     \ \ge \
     t_e + \big((z_{a,e} - z_{a,e-1} ) - (z_{a,f} - z_{a,f-1}) - 1\big) p_a
     \qquad\text{for all } f>e\in E,\; a\in A
  \end{equation}
  As discussed in \cite{artigues-etal11}, these constraints ensure that,
if activity $a$ starts at event $e$ and ends at~$f$, then the time difference between  $f$
and $e$ is at least the processing time of $a$: $t_f \ge t_e + p_a$.

\item[Contiguity constraints] As proved in \cite[Proposition 1]{artigues-etal11a}, the
  constraints
  \begin{eqnarray}
    \sum_{i=1}^{e-1} z_{a,i}
    &\le&
    \phantom{(n-e)}\llap{$e$}\,\big(1-(z_{a,e} - z_{a,e-1})\big)
    \qquad\text{for all } e\in E\smallsetminus\{1\},\; a\in A
    \\
    \sum_{i=e}^n z_{a,i}
    &\le&
    (n-e)\,\big(1+(z_{a,e} - z_{a,e-1})\big)
    \qquad\text{for all } e\in E\smallsetminus\{1\},\; a\in A
  \end{eqnarray}
  ensure \emph{non-preemption}, i.e., the events after which the activity $a$ is being
  processed are adjacent.

\item[Precedence constraints] The implication $(z_{a,e}=1) \Longrightarrow
  \big(\sum_{i=1}^{e} z_{b,i}=0\big)$ that describes the directed edge $a\lra b\in H$ for
  each event~$e$ is modeled by the linear inequality
  \begin{equation}
     z_{a,e} + \sum_{i=1}^e z_{b,i}
     \ \le \
     1+(1-z_{a,e})e
     \qquad\text{for all } e\in E,\; a\lra b\in H
  \end{equation}

% \item[Space constraints] At any given time, all current activities must take up no more
%   than the entire available space:
%   \begin{equation}
%     \sum_{e=1}^n s_a \, z_{a,e} 
%     \ \le \
%     S \qquad\text{for all } a\in A
%   \end{equation}
\end{mydesc}

\subsection{External constraints}

We may also incorporate constraints that come from the way crops behave. For example,
Gilad states that \emph{``A head of lettuce planted in summer must be harvested
  the week after it is planted, but if it is planted in winter, it can stay in the ground
  for up to two months.''} This can be modeled via a sequence of chains 
\begin{eqnarray*}
   \chi_{\text{lettuce},\, 25} 
   &=& 
   \big(
      pl_{\text{lettuce},\, 25}, \; 
      we_{\text{lettuce},\, [25,26]}, \;
      gr_{\text{lettuce},\, [25,26]}, \;
      ha_{\text{lettuce},\, [25,26]}
      \big), \\
   \chi_{\text{lettuce},\, 26} 
   &=& 
   \big(
      pl_{\text{lettuce},\, 26}, \; 
      we_{\text{lettuce},\, [26,27]}, \;
      gr_{\text{lettuce},\, [26,27]}, \;
      ha_{\text{lettuce},\, [26,27]}
      \big), \\
   &\dots& \\
   \chi_{\text{lettuce},\, 35} 
   &=& 
   \big(
      pl_{\text{lettuce},\, 35}, \; 
      we_{\text{lettuce},\, [35,36]}, \;
      gr_{\text{lettuce},\, [35,36]}, \;
      ha_{\text{lettuce},\, [35,36]}
      \big)
\end{eqnarray*}
that say that lettuces \emph{planted} from the middle of June (week 25) to the last week
of August (week 35) must be \emph{weeded} exactly once, need one week to \emph{grow}, and
must be \emph{harvested} one week after planting; and a sequence of chains
\begin{eqnarray*}
    \chi_{\text{lettuce},\, 47} 
   &=& 
   \big(
      pl_{\text{lettuce},\, 47}, \; 
      we_{\text{lettuce},\, [47,49]}, \;
      we_{\text{lettuce},\, [50,52]}, \;
      we_{\text{lettuce},\, [53,55]}, 
   \\ && \hspace{6cm}
      gr_{\text{lettuce},\, [47,55]}, \;
      ha_{\text{lettuce},\, [47,55]} \;
      \big), \\
    \chi_{\text{lettuce},\, 48} 
   &=& 
   \big(
      pl_{\text{lettuce},\, 48}, \; 
      we_{\text{lettuce},\, [48,50]}, \;
      we_{\text{lettuce},\, [51,53]}, \;
      we_{\text{lettuce},\, [54,56]}, 
   \\ && \hspace{6cm}
      gr_{\text{lettuce},\, [48,56]}, \;
      ha_{\text{lettuce},\, [48,56]} \;
      \big), \\
   &\dots& \\
   \chi_{\text{lettuce},\, 56} 
   &=& 
   \big(
      pl_{\text{lettuce},\, 56}, \; 
      we_{\text{lettuce},\, [56,58]}, \;
      we_{\text{lettuce},\, [59,61]}, \;
      we_{\text{lettuce},\, [62,64]}, 
   \\ && \hspace{6cm}
      gr_{\text{lettuce},\, [56,64]}, \;
      ha_{\text{lettuce},\, [56,64]} \;
      \big)
\end{eqnarray*}
that express that if a head of lettuce is planted between the third week of November (week 47)
and the last week of January (week 56), up to eight weeks can pass before it must be
harvested ($56+8=64$); in exchange for that, we must weed three times. 


\subsection{Objective function}

The objective function we want to  maximize is thus
\[
   f 
   \ = \
   \sum_{c\in C, y\in Y} y_{c,w} \sum_{a\in A} x_{c,w,a}
\]


\section{Allocation}


\section{Implementing and optimizing the problem formulation}

\subsection{Implementation}

We will probably use either PHP or Python to generate the input file to the optimizer from
a database of constraints and other data.

\subsection{Optimizations}
As remarked in \cite{artigues-etal11}, there is no need to create
events for the ending of the last activities. 

\section{Server-side technology}

Gilad's intention is to make the program available on a server. That's
fine, except that we need to be able to install c++ and cbc on such a server.

\bibliographystyle{siam}
\bibliography{specification}

\end{document}
