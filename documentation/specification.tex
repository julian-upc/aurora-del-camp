\documentclass[11pt]{amsart}

\usepackage{times}
\usepackage{microtype}
\usepackage{amssymb}
\usepackage{paralist}
\usepackage{url}
\usepackage{a4wide}

\numberwithin{equation}{section}
\newcommand{\ojo}[1]{\textsf{\bfseries\boldmath{#1}}}

\begin{document}

\renewcommand*\descriptionlabel[1]{%
\hspace\labelsep\normalfont\itshape #1:}.

\title{Aurora del Camp -- specification}
\author{Julian Pfeifle}
\date{Version of \today}
\maketitle

This document outlines some of the issues that arise in the
implementation of Gilad's proposal~\cite{buzi11}.

\medskip
Given the availability of powerful free and open source solvers for
integer programs such as CBC~\cite{cbc}, it seems natural to pursue an
integer programming formulation. Of course, free solvers are not as
good as the best commercial ones, but the most recent
benchmarks~\cite{mittelmann11} indicate that CBC is reasonably
competitive; more precisely, it's the most competitive among all
solvers that have an open source license (in the case of CBC, the
``Eclipse Public License'') that permits Gilad to use it commercially
without paying any license fees.

\section{Event-based modeling}

We first list the activities involved.

\subsection{Tasks to be considered}

The tasks relevant to the problem come in two types, those that only
affect the field, and are thus common to all crops, and those that are
specific to each crop.

\begin{description}
\item[Tasks common to all crops]  $ti$~for tilling, $rv$~for
  rotovating, $gm$~for green manure planting, $ft$~for fertilizing,
  $bb$~for bed building, $si$~for setting up irrigation, $sr$ for
  setting rows, $we$ for weeding.

\smallskip
\item[Tasks specific to a crop] $by$ for buying seeds, $ss$ for
  soaking seeds, $cs$ for cutting or separating cloned seeds, $gc$~for
  false germination and cleaning, $pl$~for planting\footnote{We
    consider transplanting and planting to be the same process.},
  $f\!u$ for fumigating, $th$ for thinning, $tr$ for trimming, $co$
  for covering, $ha$~for harvesting.
\end{description}

We record the precedence constraints between these tasks in a directed
acyclic graph~$E$. Thus, $t_1\longrightarrow t_2$ (also written
$t_1<t_2$) is a directed edge in~$E$ if $t_1$~must be completed
before $t_2$~can start.

\subsection{Events}


We separate the \emph{scheduling} part of the problem, in which the
appropriate sequencing of events is determined, from the
\emph{allocation} part, in which tasks and crops are assigned their
proper place in the field. First, the optimal sequencing of events is
determined in a way that respects the available field space and work
force using an \emph{event-based} formulation (see below); in a second
step, each task is allocated its proper space in the field. Separating
the two phases makes it easier to formulate each one, and presumably
makes them (and thus, the whole problem) easier to solve.


Here we follow \cite{artigues-etal11}. Put briefly, the resources
consumed are time, money and space in the fields, and money is
produced again by selling the crops.

In general terms, we will plan over several years.

As in \cite{artigues-etal11}, we index the set of tasks as
$T=\{t_i:i\in I\}$, using an index set $I=\{1,2,\dots,n\}$. The idea
is to have ``planting a batch of tomatoes in summmer 2011'',
``planting another batch of tomatoes in summer 2011'', and ``planting
a third batch of tomatoes, but in the summer of 2012'' to be different
tasks. This allows Gilad to 



\subsection{Objective function}

Each crop $c\in C$ has a yield of  $y_{c,w}$, depending on the week
$w\in W$ it is planted. 
The objective function we want to  maximize is thus
\[
   f 
   \ = \
   \sum_{c\in C, y\in Y} y_{c,w} \sum_{a\in A} x_{c,w,a}
\]

\section{Server-side technology}

Gilad's intention is to make the program available on a server. That's
fine, except that we need to be able to install c++ and cbc on such a server.

\bibliographystyle{siam}
\bibliography{specification}

\end{document}
