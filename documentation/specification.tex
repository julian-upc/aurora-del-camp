\documentclass[11pt]{amsart}

\usepackage{times}
\usepackage{amssymb}
\usepackage{paralist}
\usepackage{url}
\usepackage{a4wide}

\newcommand{\ojo}[1]{\textsf{\bfseries\boldmath{#1}}}

\begin{document}

\renewcommand*\descriptionlabel[1]{%
\hspace\labelsep\normalfont\itshape #1:}.

\title{Aurora del Camp -- specification}
\author{Julian Pfeifle}
\date{Version of \today}
\maketitle

This document outlines some of the issues that arise in the
implementation of Gilad's proposal~\cite{buzi11}.

\medskip
Given the availability of powerful free and open source solvers for
integer programs such as CBC~\cite{cbc}, it seems natural to pursue an
integer programming formulation. Of course, free solvers are not as
good as the best commercial ones, but the most recent
benchmarks~\cite{mittelmann11} indicate that CBC is reasonably
competitive; more precisely, it's the most competitive among all
solvers that have an open source license (in the case of CBC, the
``Eclipse Public License'') that permits Gilad to use it commercially
without paying any license fees.

\section{Problem formulation}

We start by translating Gilad's document into this language.

\subsection{Variables}

We work with a set $C$ of crops, the set $W=\{1,\dots,52\}$ of weeks
in the year, and a set $A$ of ``unit acres'', i.e., indivisible units
of farmland dedicated to a single crop or task. The set $A$ has a
distinguished member $a_0\in A$ that stands for off-field work. In
this fictitious area of acre, any number of tasks may be performed
simultaneously, while only one thing at a time may be done in all
acres $A\smallsetminus a_0$. 

We will have several different families of variables, one family for
each specific task. These are further subdivided into families that
only affect the field, and are thus common to all crops, and those
that aer specific to each crop.

\begin{description}
\item[Tasks common to all crops]  $ti$~for tilling, $rv$~for
  rotovating, $gm$~for green manure planting, $ft$~for fertilizing,
  $bb$~for bed building, $si$~for setting up irrigation, $sr$ for
  setting rows. 

\noindent
  Each of these families contains variables indexed by the week~$w\in
  W$ of the year, and the piece of land~$a\in A$.

\item[Tasks specific to a crop] $ss$ for soaking seeds, $cs$ for
  cutting or separating cloned seeds, $gc$~for false germination and
  cleaning, $pl$~for planting\footnote{\ojo{Is transplanting and planting
    the same process?}}, $we$ for weeding, $fu$ for fumigating, $th$
for thinning, $tr$ for trimming, $co$ for covering, $ha$~for harvesting.\footnote{\ojo{I
      left out buying seeds, because that seems easy enough to do. Is
      that all right?}}

\noindent
  Each of these families contains variables indexed by the type of
  crop $c\in C$, the week $w\in W$ of the year, and the piece of land
  $a\in A$.
\end{description}

Each variable can take on the value~$0$ or~$1$, depending on whether
or not the task at hand is undertaken for crop~$c$ in the unit
acre~$a$ during week~$w$.  For instance, the set of seed-soaking
variables is $ss=\{ss_{c,w,a} : c\in C, w\in W, a\in A\}$. The
families common to all crops have less variables: for instance, the
tilling variables are just $ti=\{ti_{w,a}:w\in W, a\in A\}$.

With an estimated $40$ types of crops and $40$ unit acres, we get
\[
   7\times 40\times 52 \text{ (common) }
   \ + \ 
   10\times40\times 40\times 52 \text{ (specific)}
   \ = \ 
   846\,560 \text{ variables.}
\]
This is well within the reach of commercial solvers such as
Gurobi; we'll have to see how cbc performs.

\subsection{Constraints}

Some of the more obvious constraints are the following:

\begin{description}
\item[In each week, there is only one crop at each unit acre] 
\[
   \sum_{c\in C} x_{c,w,a} \ = \ 1
   \qquad\text{for all } w\in W \text{ and } a\in A
\]

\ojo{Finish the constraints, starting with the precedence constraints}
\end{description}


\subsection{Objective function}

Each crop $c\in C$ has a yield of  $y_{c,w}$, depending on the week
$w\in W$ it is planted. Since the default direction in optimization
appears to be minimization, we make the yields negative.

The objective function is thus
\[
   f 
   \ = \
   -\sum_{c\in C, y\in Y} y_{c,w} \sum_{a\in A} x_{c,w,a}
\]

\section{Server-side technology}

Gilad's intention is to make the program available on a server. That's
fine, except that we need to be able to install c++ and cbc on such a server.

\bibliographystyle{siam}
\bibliography{specification}

\end{document}
