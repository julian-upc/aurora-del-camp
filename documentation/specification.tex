\documentclass[11pt]{amsart}

\usepackage{times}
\usepackage{microtype}
\usepackage{amssymb}
\usepackage{paralist}
\usepackage{url}
\usepackage{a4wide}

\numberwithin{equation}{section}
\newcommand{\ojo}[1]{\textsf{\bfseries\boldmath{#1}}}

\begin{document}

\renewcommand*\descriptionlabel[1]{%
\hspace\labelsep\normalfont\itshape #1:}.

\title{Aurora del Camp -- specification}
\author{Julian Pfeifle}
\date{Version of \today}
\maketitle

This document outlines some of the issues that arise in the
implementation of Gilad's proposal~\cite{buzi11}.

\medskip
Given the availability of powerful free and open source solvers for
integer programs such as CBC~\cite{cbc}, it seems natural to pursue an
integer programming formulation. Of course, free solvers are not as
good as the best commercial ones, but the most recent
benchmarks~\cite{mittelmann11} indicate that CBC is reasonably
competitive; more precisely, it's the most competitive among all
solvers that have an open source license (in the case of CBC, the
``Eclipse Public License'') that permits Gilad to use it commercially
without paying any license fees.

\section{Approaches to modeling}

Before summarizing some  approaches to modeling the problem,
we briefly list the activities involved.

\subsection{Tasks to be considered}

The tasks relevant to the problem come in two types, those that only
affect the field, and are thus common to all crops, and those that are
specific to each crop.

\begin{description}
\item[Tasks common to all crops]  $ti$~for tilling, $rv$~for
  rotovating, $gm$~for green manure planting, $ft$~for fertilizing,
  $bb$~for bed building, $si$~for setting up irrigation, $sr$ for
  setting rows, $we$ for weeding.

\smallskip
\item[Tasks specific to a crop] $by$ for buying seeds, $ss$ for
  soaking seeds, $cs$ for cutting or separating cloned seeds, $gc$~for
  false germination and cleaning, $pl$~for planting\footnote{We
    consider transplanting and planting to be the same process.},
  $f\!u$ for fumigating, $th$ for thinning, $tr$ for trimming, $co$
  for covering, $ha$~for harvesting.
\end{description}

We record the precedence constraints between these tasks in a directed
acyclic graph~$E$. Thus, $t_1\longrightarrow t_2$ (also written
$t_1<t_2$) is a directed edge in~$E$ if $t_1$~must be completed
before $t_2$~can start.

\subsection{Approaches}

We consider two types of models:

\begin{description}
\item[Discrete time, all-in-one] The first model discretizes time into
  52~weeks, and reserves a 0/1~variable for each triple in $T_c\times
  W\times A$, respectively quadruple in $T_s\times C\times W\times
  A$. Here $T=T_c\cup T_s$ denotes the set of common and specific
  tasks, $C$~the set of crops, $W$~the set of weeks in the planning
  horizon of the model (so, for example, $W=\{0,1,\dots,51\}$
  corresponds to modelling one year), and $A$~the set of \emph{unit
    lots} that the available farmland is divided into.  Each variable,
  say $v_{t,w,a}$, takes the value~$1$ (respectively,~$0$), if the
  task~$t$ is (respectively, is~not) executed in week~$w$ in the unit
  lot~$a$. Thus, this model simultaneously solves the
  \emph{scheduling} part of the problem, in which the appropriate
  sequencing of events is determined, and the \emph{allocation} part,
  in which tasks and crops are assigned their proper place in the
  field.

\item[Event-based, separate] The second model separates these two
  parts of the problem. First, the optimal sequencing of events is
  determined in a way that respects the available field space and work
  force using an \emph{event-based} formulation (see below); in a
  second step, each task is allocated its proper space in the
  field. Separating the two phases makes it easier to formulate each
  one, and presumably makes them (and thus, the whole problem) easier
  to solve.
\end{description}

Since the event-based strategy appears to be more powerful, we
describe it first; the discrete-time formulation might eventually
disappear completely from this document.

\section{Problem formulation using events}

Here we follow \cite{artigues-etal11}. Put briefly, the resources
consumed are time, money and space in the fields, and money is
produced again by selling the crops.

In general terms, we will plan over several years.

As in \cite{artigues-etal11}, we index the set of tasks as
$T=\{t_i:i\in I\}$, using an index set $I=\{1,2,\dots,n\}$. The idea
is to have ``planting a batch of tomatoes in summmer 2011'',
``planting another batch of tomatoes in summer 2011'', and ``planting
a third batch of tomatoes, but in the summer of 2012'' to be different
tasks. This allows Gilad to 

\section{Problem formulation using discrete time}

This kind of formulation needs significantly more variables, but has a
better LP-relaxation. Only by testing both will we be able to
determine which is the superior method for our problem. To keep the
number of variables somewhat managable, we work with \emph{circular time} by
discretizing one year into 52 weeks, and identifying the start and end
of that year.
  
\subsection{Variables}

We work with a set $C$ of crops, the set $W=\{0,\dots,51\}$ of weeks
in the year, and a set $A$ of ``unit lots'', i.e., indivisible units
of farmland dedicated to a single crop or task. The set $A$ has a
distinguished member $a_0\in A$ that stands for off-field work. In
this fictitious lot, any number of tasks may be performed
simultaneously, while only one thing at a time may be done in all lots
$A\smallsetminus a_0$.

In the discrete time formulation, each of the ``common'' families
contains variables indexed by the week~$w\in W$ of the year, and the
piece of land~$a\in A$. For instance, the tilling variables are
$ti=\{ti_{w,a}:w\in W, a\in A\}$. The ``crop-specific'' families, on
the other hand, contain more variables, because they also depend on
the type of crop $c\in C$. For example, the set of seed-soaking
variables is $ss=\{ss_{c,w,a} : c\in C, w\in W, a\in A\}$. Each
individual variable in each of these families can take on the
value~$0$ or~$1$, depending on whether or not the task at hand is
undertaken for crop~$c$ in the unit lot~$a$ during week~$w$.

With an estimated $40$ types of crops and $40$ unit lots, we get an
upper bound of
\[
   8\times 40\times 52 \text{ (common) }
   \ + \ 
   10\times 40\times 40\times 52 \text{ (specific)}
   \ = \ 
   848\,640 \text{ variables.}
\]
One the one hand, this is well within the reach of commercial solvers
such as Gurobi (we'll have to see how cbc performs, though); on the
other, there will actually be substantially less variables than this
because not all crops need all variables. For example, a crop that is
bought as a seedling from a nursery does not need variables from the
families $by, ss, cs, gc$.

\subsection{Constraints}

To formulate our constraints, we need some additional data on our
crops:
\begin{description}
\item[Plantation interval $pi_{c}$] How many weeks must pass, at
  least, between plantings of crop~$c$ 

\item[Yield $y_{c,i}$] The amount of fruit that crop $c$ yields
  $i$~weeks after planting, for $0\le i\le pi_c$.
\end{description}

Now  we can formulate the constraints:

\begin{enumerate}
\item \emph{In each week, there is at most one crop planted at each unit lot:}
\[
   \sum_{c\in C} pl_{c,w,a} \ \le \ 1
   \qquad\text{for all } w\in W \text{ and } a\in A
\]

\item\emph{Respect plantation intervals:}
\begin{equation}\label{eq:pi}
   \sum_{i=w}^{w+pi_c} pl_{c,i,a} \ \le \ 1
   \qquad\text{for all } w\in W, c\in C, a\in A,
\end{equation}
because during any interval of $pi_c$ consecutive weeks, there may
occur at most one planting.  Here and throughout, index additions such
as $i+pi_c$ are understood to be taken modulo 52.

\smallskip
\item\emph{Prescribe the yield:} If we want the total yield of crop~$c$ in
  a given week~$w\in W$ to be $Y_{c,w}$, we must impose
\[
    \sum_{a\in A} \sum_{i=0}^{pi_c} y_{c,i}\; pl_{c,w-i,a}
    \ = \
    Y_{c,w},
\]
because a crop planted $i$ weeks before week $w$ has a yield of
$y_{c,i}$ in week~$w$.  This works because by the foregoing
constraint~\eqref{eq:pi}, at most one of the variables
$\{pl_{c,w-i,a}:0\le i\le pi_c\}$ can have the value~$1$, while the rest must
be~$0$.

\smallskip
\item\emph{Precedence constraints:} Here we must deal with the fact
  that we model our year to be cyclic, i.e., the same 52~weeks repeat
  again and again, as witnessed by our index addition modulo~52 in
  condition~\eqref{eq:pi}. Nevertheless, some tasks must be completed
  before others can start, which implies a linear and not circular
  ordering of time. We model this by interpreting precedence
  constraints to be valid only within a half-year horizon, so
  that \emph{``$x$ must happen before $y$''} is interpreted as meaning
  \emph{``$x$ may not happen until half a year after $y$''}.
  For example, the constraint that ``tilling'' must happen before ``weeding''
  for each crop is modeled as
  \begin{align*}
     \textstyle ti_{w+i,a} & \ \le \  we_{w,a} &
     \text{for all } w\in W,\ a\in A, \text{ and }
     i\in\{0,\dots,26\},\\
     \textstyle ti_{w+i,a} + we_{w,a} & \ \le\   1 &
     \text{for all } w\in W,\ a\in A, \text{ and }
     i\in\{0,\dots,26\}.
  \end{align*}
  To check this, note that these inequalities are not satisfied, for
  example, if $ti_{4,c}=1$ and $we_{0,c}=1$ (which is as it should be,
  because these variables say we till in week~4, after having weeded
  in week~0), but they are satisfied if $ti_{0,c}=1$ and $we_{4,c}=1$.

\end{enumerate}


\subsection{Objective function}

Each crop $c\in C$ has a yield of  $y_{c,w}$, depending on the week
$w\in W$ it is planted. 
The objective function we want to  maximize is thus
\[
   f 
   \ = \
   \sum_{c\in C, y\in Y} y_{c,w} \sum_{a\in A} x_{c,w,a}
\]

\section{Server-side technology}

Gilad's intention is to make the program available on a server. That's
fine, except that we need to be able to install c++ and cbc on such a server.

\bibliographystyle{siam}
\bibliography{specification}

\end{document}
