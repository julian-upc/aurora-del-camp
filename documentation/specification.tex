\documentclass[11pt]{amsart}

\usepackage{times}
\usepackage{microtype}
\usepackage{amssymb}
\usepackage{paralist}
\usepackage{url}
\usepackage{a4wide}
\usepackage{bbm}

\newcommand{\NN}{\mathbbm{N}}
\newcommand{\ZZ}{\mathbbm{Z}}

\newcommand{\lra}{\longrightarrow}

\numberwithin{equation}{section}
\newcommand{\ojo}[1]{\textsf{\bfseries\boldmath{#1}}}

\begin{document}

\renewcommand*\descriptionlabel[1]{%
\hspace\labelsep\normalfont\itshape #1:}

\title{Aurora del Camp -- specification}
\author{Julian Pfeifle}
\date{Version of \today}
\maketitle

This document outlines some of the issues that arise in the
implementation of Gilad's proposal~\cite{buzi11}.

\medskip
Given the availability of powerful free and open source solvers for
integer programs such as CBC~\cite{cbc}, it seems natural to pursue an
integer programming formulation. Of course, free solvers are not as
good as the best commercial ones, but the most recent
benchmarks~\cite{mittelmann11} indicate that CBC is reasonably
competitive; more precisely, it's the most competitive among all
solvers that have an open source license (in the case of CBC, the
``Eclipse Public License'') that permits Gilad to use it commercially
without paying any license fees.

\section{Event-based modeling}


\subsection{Activities to be considered}

In the parlance of \cite{artigues-etal11}, we want to schedule a set
of \emph{activities} subject to certain constraints. In our setup,
each activity is associated to a certain specific part of the
available fields. We refer to these parts or areas as \emph{lots}, and
collect them into a set~$L$. We allow them to have different
sizes.

The activities to be carried out in or on these lots come in two types: $A=A_c\cup A_s$,
where the activities in $A_c$ only affect the field and are thus \emph{``common''} to all
crops, and the activities~$A_s$ are \emph{``specific''} to each crop. Moreover, we allow
each activity to occur multiple times, to take into account repeated sowing, harvesting,
etc, and collect the indices of the possible repetitions into a set $R\subset\NN$.  Gilad
breaks these two types of activities down as follows:

\begin{description}
\item[Activities common to all crops]  $ti$~for tilling, $rv$~for
  rotovating, $gm$~for green manure planting, $f\!t$~for fertilizing,
  $bb$~for bed building, $si$~for setting up irrigation, $sr$ for
  setting rows, $we$ for weeding. Since each of these activities can
  occur on each lot, and be repeated, we set
  \[
     A_c 
     \ = \
     \{ ti,\;rv,\;gm,\;f\!t,\;bb,\;si,\;sr,\;we\} \times L \times R.
  \]
  A typical activity in $A_c$ is therefore $(ti,\ell, i)$ for some
  $\ell\in L$ and $i\in R$, which we write as $ti_{\ell, i}$ and take
  to mean ``tilling the lot $\ell$ for the $i$-th time''.

\smallskip
\item[Activities specific to a crop] $by$ for buying seeds, $ss$ for
  soaking seeds, $cs$ for cutting or separating cloned seeds, $gc$~for
  false germination and cleaning, $pl$~for planting\footnote{We
    consider transplanting and planting to be the same process.},
  $f\!u$ for fumigating, $th$ for thinning, $tr$ for trimming, $co$
  for covering, $ha$~for harvesting. We assemble these into
  \[
     A_s = \{ by,\;ss,\;cs,\;gc,\;pl,\;f\!u,\;th,\;tr,\;co,\;ha\}
     \times C \times L \times R,
  \]
  where $C$ is the set of crops. A typical activity in~$A_s$ is therefore
  $(ha,c,\ell, i)=ha_{c,\ell, i}$, which means ``harvesting the crop $c\in
  C$ in the lot $\ell\in L$ for the $i$-th time''.
\end{description}

The estimates $|L|=30$, $|R|=5$, $|C|=40$ yield an upper bound of
\[
    8 \times 30 \times 5 + 10 \times 40 \times 30 \times 5
    \ = \
    61\,200
\]
activities in the model, which is a very manageable figure for
commercial solvers, and should also present few problems to free
solvers such as CBC.


\subsection{Precedence constraints}

We record the precedence constraints between these activities in a directed acyclic
graph~$H$. Thus, $t_1\lra t_2$ (also written $t_1<t_2$) is a directed edge
in~$H$ if $t_1$~must be completed before $t_2$~can start. Some sample precedence
constraints are the following:

\begin{description}
\item[Earlier repetitions execute before later ones] Thus, $x_{c,\ell,i}\lra
  x_{c,\ell,j}$ is an edge in~$H$ if $i<j$.
\end{description}

\subsection{Overview of the model}


We separate the \emph{scheduling} part of the problem, in which the
appropriate sequencing of events is determined, from the
\emph{allocation} part, in which activities and crops are assigned
their proper place in the field. First, the optimal sequencing of
events is determined in a way that respects the available field space
and work force using an \emph{event-based} formulation (see below);
allocation is relegated to a second step. Separating the two phases
makes it easier to formulate each one, and presumably makes them (and
thus, the whole problem) easier to solve.

In general terms, we will plan over several years. In the final
program, there will be an interface to put new activities into a
\emph{task queue} (a set of events that is not scheduled yet), and an
interface to record the actual progress of activities. This act of
recording the real start and end time of activities, environmental
changes, changes in the available work force, etc., sets certain
variables in the problem formulation to fixed, known values, and
allows Gilad to frequently update the solution of the optimization
problem and dynamically take into account the latest developments.


\section{Scheduling}

Here we follow \cite{artigues-etal11}. Put briefly, the main
\emph{non-renewable} resource consumed is ``time'', while some of the
\emph{renewable} ones are ``space in the fields'', ``money'' and
``gasoline''. Money is put back into the system by selling the crops,
and field space by tilling the remains of the crop. Gasoline is
renewable, but costs money.

To the activities in $A$ we associate a set~$E$ of \emph{events},
which consist of the acts of starting and finishing each activity
in~$A$; thus, $|E|=2|A|=:n$. In consequence, we may consider
$E=\{1,2,\dots, 2|A|=n\}$ to be totally ordered.  Following
\cite{artigues-etal11}, we introduce the following variables:

\begin{enumerate}
\item A set of binary decision variables 
  \[
     Z
     \ = \
     \big\{z_{a,e}: a\in A, e\in E\big\},
  \]
  where each $z_{a,e}=1$ if and only if activity~$a$ starts at
  event~$e$ or is still in execution at event~$e$.

\smallskip
\item A set of continuous variables
  \[
     T
     \ = \
     \big\{t_e : e\in E\big\}, 
  \]
  that indicate the starting time of each event.
\end{enumerate}

We now adapt the individual constraints from~\cite{artigues-etal11}:

\renewcommand{\arraystretch}{3}
\begin{description}

\item[Not all activities have to execute] We do \emph{not} incorporate a constraint
  $\sum_{e\in E} z_{a,e}=1$ for all $a\in A$, because we do not wish to require all
  activities to execute. This leaves Gilad margin to queue activities such as ``sowing and
  harvesting beans for the fifth time'' that may or may not take place, but where the
  decision on them having take place or not is an outcome of the optimization process and
  not an a-priori input to the problem formulation.

  Initially, it therefore seems to be a good idea to queue more repetitions of activities
  than could reasonably be undertaken, so that the optimal number of repetitions may be
  learned from the optimization process. We'll see how this works out.

\item[Setting the starting time] Instead of using $t_0=0$, we set
  \[
     t_0
     \ = \
     w_0,
  \]
  where $w_0$ indexes the week of the year where optimization starts. In general, using
  weeks as units for time seems to be a good idea.

\item[Ordering the execution starts]
  \[
     t_{e+1} 
     \ \ge \
     t_e
     \qquad\text{for all }
      e\in E\text{ with }
      e\ne n-1
  \]

\item 
\end{description}

\subsection{Objective function}

Each crop $c\in C$ has a yield of  $y_{c,w}$, depending on the week
$w\in W$ it is planted. 
The objective function we want to  maximize is thus
\[
   f 
   \ = \
   \sum_{c\in C, y\in Y} y_{c,w} \sum_{a\in A} x_{c,w,a}
\]


\section{Allocation}


\section{Implementing and optimizing the problem formulation}

\subsection{Implementation}

We will probably use either PHP or Python to generate the input file to the optimizer from
a database of constraints and other data.

\subsection{Optimizations}
As remarked in \cite{artigues-etal11}, there is no need to create
events for the ending of the last activities. 

\section{Server-side technology}

Gilad's intention is to make the program available on a server. That's
fine, except that we need to be able to install c++ and cbc on such a server.

\bibliographystyle{siam}
\bibliography{specification}

\end{document}
